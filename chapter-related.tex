\chapter{Related Work}
\label{chp:related}
\index{Related Work@\emph{Related Work}}%

This chapter presents related work. Section~\ref{sec:rel_cs} presents the work related to compressive sensing, Section~\ref{sec:rel_diagnosis} describes the work in general network diagnosis and finally Section~\ref{sec:rel_anomaly_det} talks about anomaly detection. 

\section{Compressive Sensing}
\label{sec:rel_cs}

LENS belongs to the realm of {\em compressive sensing}, a generic
methodology for extracting and exploiting the presence of certain
types of structure and redundancy in data from many real-world
systems. Compressive sensing has recently attracted considerable
attentions from statistics, approximation theory, information theory,
and signal processing
\cite{candes06:_compressive,donoho06:_compr_sensin,candes:_exact_matrix,recht:_guaran,recht08:_nec,zhang09sensing}
and is rapidly becoming a vibrant research area of its own.

Most existing compressive sensing works assume that the matrices
satisfy low-rank property. However, this assumption may not hold. 
Violation of such assumption significantly limits the accuracy of 
these techniques. 

Significant work has been done for solving
under-determined linear inverse problems. Missing value interpolation,
prediction, and network tomography can be cast into the same formulation.

As described in \cite{ZRLD03}, many solutions solve the
regularized least-squares problem:
$\min_{\bx} \|\by-A\bx\|_2^2 + \lambda^2 J(\bx)$,
where $\|\cdot\|_2$ denotes the $L_2$ norm, $\lambda$ is a
regularization parameter, and $J(\bx)$ is a penalization functional.
In $L_2$ norm minimization, which is a widely used solution to linear
inference problem, $J(\bx) = \|\bx\|_2^2$. In $L_1$ norm minimization,
another commonly used scheme, $J(\bx) =
\|\bx\|_1$ (\ie, the $L_1$ norm of $\bx$). 


Other regularization terms include $\|X\|_*$, the nuclear norm of
matrix $X$, and spatio-temporal terms $\|SX\|_F^2$ and $\|XT^T\|_F^2$
in \cite{zhang09sensing}. Unique advantages of LENS formulation
include (i) its
general formulation to account for a low-rank component, a sparse anomaly
component, a dense but small noise
term, and domain knowledge, (ii) its effective optimization
algorithm to solve the general decomposition problem, and (iii) a
data-driven procedure to learn the parameters. Its formulation is more
general than existing work (\eg, \cite{zhang09sensing}) in that
\cite{zhang09sensing} requires the original matrix to be well approximated by
  the product of two low-rank matrices, whereas LENS relaxes this
  constraint and allows the delta between the original matrix and
  sparse anomaly matrix to be approximated by the product of two
  low-rank matrices. Moreover, LENS allows a linear
  coefficient in each of the decomposed terms and supports general
  forms of domain knowledge.


\section{Network Diagnosis}
\label{sec:rel_diagnosis}

There has been significant work on anomaly detection and network
diagnosis. PCA (\eg, ~\cite{PCA1,PCA2,PCA3}) has been widely used for anomaly
detection. \cite{PCA-sensitivity} shows that PCA is sensitive to how
many principal components are used. \cite{PCA-data-poison} shows that
data poisoning can significantly degrade the performance of PCA.
%and proposes using a robust Laplace cut-off threshold to improve its
% robustness.
Barford et al.~\cite{barford-wavelet} uses wavelets to decompose
an original signal into low-, mid-, and high-frequency
components and then detect anomalies based on the high-frequency
components. Zhang et al.~\cite{zhang09sensing} uses
compressive sensing to discover anomalies in traffic matrices.
\cite{anomography} develops a framework to capture a
range of detectors. 
% \cite{} uses risk modeling to map high-level failure
% notifications into lower-layer root causes.
% WISE~\cite{WISE} presents
% a what-if analysis tool to estimate the effects of network
% configuration changes on service response times.   %\cite{Q-score} 
\cite{PRISM} proposes a multi-scale robust subspace algorithm to
identify changes in performance even when the baseline is
contaminated.  \cite{Q-score} uses ridge regression to learn Quality of
Experience (QoE) in large-scale IPTV systems. Ridge regression does
not work well in our context due to possible under-constraint
issues and varying customers' response time. 
NICE~\cite{NICE} uses
statistical correlation to detect chronic network
problems. Mercury~\cite{Mercury} detects persistent behavior changes
using the time-alignment for distributed
triggers.
\cite{anomaly-icdcs11} combines different anomaly
detection methods, such as EWMA, FFT, Holt-Winters, and Wavelets, to boost
the performance. 

\cite{Tiresias} uses both customer call dataset and network
crash log in IPTV to detect anomalies. It first finds heavy hitters
where there might be anomalies with high probability and then use EWMA
to detect anomalies.

The statistical and multi-scale analysis used in the previous works are
complementary to our work. We complement the previous works by using
$L_1$ minimization to select important metrics, leveraging temporal
stability and low rank to enhance the accuracy of regression, and
applying multiple classifiers to enhance robustness. These techniques
can be potentially useful to other network diagnosis problems.

\para{Event detection using Twitter:}
Twitter has been extensively studied to detect various events % due to its real-time property and massive scale. 
such as service issues~\cite{listentome, online-services}, earthquakes~\cite{earthquake}, 
stock markets~\cite{stock2}, elections~\cite{election10}, and public health issues~\cite{publichealth}.
\cite{listentome} shows the feasibility of social media to 
understand user experiences and finds correlation between tweets and customer tickets. 
% \cite{online-services} infers the online Internet 
% service availability by tracking the volumes of tweets with the phrase ``X is down'' or the hashtag ``Xfail''.
% They have detected outages of popular services (\eg, Gmail, Bing, PayPal). 
\cite{online-services} detects outages of popular services (\eg, Gmail, Bing, PayPal)
by tracking the volumes of tweets with the phrase ``X is down'' or the hashtag ``Xfail''.
Our work takes a step further to summarize the events and localize the impacted regions 
using detailed information from tweets. 


\section{Anomaly Detection}
\label{sec:rel_anomaly_det}

Anomaly detection has been extensively
studied. PCA (\eg, ~\cite{PCA1,PCA2,PCA3}) has been widely used for anomaly
detection. PCA has several well known limitations: it is sensitive
to how many principal components are used~\cite{PCA-sensitivity} and
vulnerable to data poisoning~\cite{PCA-data-poison}.
Barford et al.~\cite{barford-wavelet} uses wavelets to decompose
an original signal into low-, mid-, and high-frequency
components and use the high-frequency components for anomaly
detection. \cite{anomography} presents a framework that incorporates a
variety of anomaly detectors. \cite{zhang09sensing} uses SRMF for anomaly
detection based on the difference between estimated and actual matrix values.
